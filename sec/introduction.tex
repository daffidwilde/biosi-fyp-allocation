\section{Introduction}\label{sec:intro}

% TODO Introduce the project allocation problem and motivating factors
% TODO Define SA
% TODO Summarise the findings of the BIOSI allocation
% TODO Signpost the rest of the paper

For many undergraduate students, a crucial part of their degree is their
final-year project (FYP). This piece of work characterises their interests and
allows the student to demonstrate their command of a chosen subject. Being
assigned a favourable FYP topic is of great importance. Good allocation affects
the student experience, improving student-supervisor relationships, engagement,
and, eventually, satisfaction~\cite{Briffa2018,Kuh2009}.

However, as the ratio between students and university staff
increases~\cite{McDonald2013}, so does the need for fair and efficient FYP
allocation systems. This need is both logistical and pedagogic. Logistical in
that as cohort sizes increase, the demands on administrative staff grow, meaning
manual systems eventually become infeasible. Pedagogical, given the impacts of
good project allocation on learning.

FYP allocation is a resource allocation problem with a specific set of
constraints. Typically, these correspond to student preferences, supervisor
preferences, and workload capacities.

This manuscript uses the student-project allocation problem (SA) to model FYP
allocation in the School of Biosciences at Cardiff University (BIOSI).
Implementing the allocation as an instance of SA grants access to an algorithm
which produces a unique, student-optimal, mathematically fair allocation. In
doing so, the work hours required by the staff are reduced dramatically.
